\documentclass[../Main.tex]{subfiles}
\begin{document}

\section{Marco de antecendentes}

\begin{justify}
Con\ los años los fraudes realizados en exámenes por parte de estudiantes universitarios han estado presentes, por lo cual se han hecho diversas investigaciones, estudios e informes  que abordan esta problemática.
\end{justify}\par

\begin{justify}
La investigación Cheating and feeling honest: Committing and punishing analog versus digital academic dishonesty behaviors in higher education \cite{5} realizada en 2016 con el objetivo de examinar la deshonestidad académica entre estudiantes universitarios, analizó casos del mismo cometidos por estudiantes, así como las razones de los comportamientos y la severidad de las sanciones por violaciones de la integridad académica. Este estudio se basó en Pavela’s (1997) \cite{6} tipos de deshonestidad académica (engaño, plagio, fabricación y facilitación), además utilizaron “el marco motivacional para cometer un acto de deshonestidad académica” realizado por Murdock y Anderman \cite{7} y el “modelo de Mantenimiento del autoconcepto” realizado por Mazar, Amir y Ariely \cite{8} para analizar las razones de los comportamientos deshonestos de los estudiantes. Los resultados de esta investigación mostraron que la deshonestidad analógica (tradicional) era más frecuente que la deshonestidad digital, según la investigación mencionada señalan que en Israel el 95\% de estudiantes admitieron haber cometido algún tipo de deshonestidad académica, ya sea virtual o presencial, de los cuales el 60\% participó en la copia de documentos y el 60\% durante exámenes, mientras que en Korea otro estudio señala que el 69\% de estudiantes admitieron haber participado en al menos un tipo de deshonestidad académica. Como argumentación por parte de los estudiantes, indicaron que la razón más frecuente de su deshonestidad académica fue la necesidad de mantener una visión positiva de sí mismo como persona honesta a pesar de violar los códigos éticos.
\end{justify}\par

\begin{justify}
Por otra parte, la institución Defensor Universitario de la Universidad de Alcalá, Madrid - España, realizó un informe \cite{9} en junio de 2018 donde su objetivo comprende analizar el fraude  académico en los procesos de evaluación del aprendizaje, indagar en la legislación y el reglamento sobre el fraude y responder el interrogante principal: ¿Cómo contrarrestar el fraude académico en la Universidad de Alcalá?.
\end{justify}\par


\begin{justify}
La manera de abordar la problemática en dicho informe es mayormente desde los aspectos legales, normativas y reglamentos, con el informe se expone el estado de la situación, las circunstancias que la rodean, además, plantean aspecto relevantes sobre fraudes académicos como: tipología, factores favorecedores, consecuencias, medidas preventivas, medidas para detecciones,  procedimientos a seguir tras la detección.
\end{justify}\par

\begin{justify}
Diversos estudios realizados demuestran que el problema sucede de manera más regular de lo que parece, exponiendo que más del 30\% de los universitarios admiten haber realizado fraudes académicos, alrededor del 50\% afirman haber hecho copia en exámenes, así que, la intención del Defensor Universitario es sacar a la luz un problema que como se mencionó anteriormente se presenta con regularidad y acudir a normativas y/o reglamentos como un mecanismo efectivo.
\end{justify}\par

\begin{justify}
Actualmente existen sistemas que monitorean plataformas de E-learning \cite{10} donde las instituciones tienen implementado un LMS (Learning Management System), estos sistemas ofrecen herramientas para realizar evaluaciones tales como talleres, foros, tareas, ejercicios y exámenes , asimismo, brinda posibilidades de aplicación a cualquier tipo de prueba, como exámenes finales, pruebas de inglés, exámenes de admisión, entre otros. LMS brinda muchas ventajas pero consigo trae desventajas como las posibilidades de fraude o suplantación en un examen en línea. 
\end{justify}\par

\begin{justify}
Validar la identidad del estudiante cuando realiza un examen en línea es un desafío de seguridad debido a que ninguna plataforma LMS puede validar que efectivamente es el estudiante quien está realizando la prueba. Por esta razón nacen los servicios de eProctering \cite{11}, una implementación digital de Proctering que ha sido empleada en instituciones educativas, estos servicios constituyen un conjunto de herramientas de software, metodologías y protocolos usadas para monitorear la actividad del estudiante al realizar un examen en línea.
\end{justify}\par

\begin{justify}
La solución de eProctering puede ser implementadas en plataformas Moodle que son usadas en Campus virtuales,  la efectividad de estos servicios eProctering no son de especial inteŕes para este trabajo, por lo cual no se hará énfasis en ello; sin embargo, se evidencia que existe una forma de abordar la problemática desde el campo del desarrollo de software como herramienta para afrontar la copia en exámenes virtuales y que es importante mencionarla. A pesar de existir estos servicios, suelen tener un costo algo elevado para las instituciones y supone algunas cuestiones de privacidad, como consecuencia una posible presión psicológica para el estudiante al estar siendo monitoreado durante su prueba.
\end{justify}\par

\begin{justify}
Por este mismo camino de software de apoyo existen otros como Urkund \cite{12} o Turnitin \cite{13} que se usan para el mismo propósito de evitar fraudes académicos, otro como TestWe \cite{14}que sirve para tomar el control del ordenador y evitar que personas externas puedan suplantar al estudiante. Estos software utilizan sistemas que permiten comparar exámenes entre sí y sistemas que hacen comparaciones con fuentes disponibles en Internet. 
\end{justify}\par

\begin{justify}
Después de hacer un recorrido por investigaciones, estudios, herramientas de software con el fin de conocer el estado del arte, se observa que la problemática es abordada por la parte de vigilar, controlar, ,detectar, aplicar reglas y normativas, entre otras acciones; no obstante, también se observa que no se aborda la problemática por la vía de la generación de exámenes que su finalidad sea tratar de reducir fraudes académicos en exámenes, ni mucho menos por medio de un software que brinde una solución por esta vía, es por esto que se pretende desarrollar un prototipo web que brinde dicha solución generando exámenes con diferentes preguntas usando técnicas procesamiento del lenguaje natural. 
\end{justify}\par

\end{document}