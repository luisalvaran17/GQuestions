\documentclass[../Main.tex]{subfiles}
\begin{document}

\section{Marco conceptual}
\begin{justify}
\textbf{Deshonestidad académica:} Actos individuales o colectivos en que se presenta como propio el conocimiento ajeno, tales como: copia de exámenes, tareas, trabajos o proyectos, plagio de textos, sustitución de personas en los exámenes, falsificación de documentos o datos, presentación de trabajos o proyectos elaborados por terceros y cualquier tipo de acción que atente contra la honestidad académica en el Instituto. 
\end{justify}\par

\begin{justify}
\textbf{Sistemático:} Aquello que respeta o se adapta a un sistema: un conjunto ordenado o estructurado de principios o elementos que se relacionan entre sí. El término proviene del latín tardío systematĭcus, a su vez derivado del griego systēmatikós.
\end{justify}\par

\begin{justify}
\textbf{Escalabilidad:} Propiedad de aumentar la capacidad de trabajo de un programa sin alterar la capacidad de trabajo o tamaño de un sistema sin poner en riesgo el buen funcionamiento y la calidad del mismo. Cuando esto sucede, dicha propiedad recibe el nombre de “sistema escalable” o escalable.
\end{justify}

\clearpage

\end{document}