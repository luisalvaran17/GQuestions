\documentclass[../Main.tex]{subfiles}
\begin{document}

\section{Justificación del problema}


%\subsection{Justificación social}
%\begin{justify}
%Esta propuesta busca beneficiar a docentes, estudiantes y la Universidad del Valle en sí, %donde se aplicará la solución. Los docentes de inglés podrán contar con una herramienta %automatizada que genera diferentes exámenes y que intenta reducir los fraudes durante los %exámenes por parte de los estudiantes en el salón de clase.
%\end{justify}\par


\subsection{Justificación académica}
\begin{justify}
La propuesta se realiza con la motivación de aplicar los conocimientos aprendidos a lo largo de la carrera de Ingeniería de Sistemas en una problemática real, especialmente los cursos de Inteligencia artificial, Álgebra lineal, Desarrollo de software, Desarrollo de aplicaciones web, base de datos, Técnicas de pruebas de software, y otros cursos que no están directamente involucrados pero que han contribuido fuertemente a la capacidad de resolver problemas de modo sistemático.
\end{justify}\par

\subsection{Justificación económica}
\begin{justify}
El prototipo web será desarrollado para ser ejecutado en la Universidad del Valle sede Tuluá para docentes de inglés, contar con una herramienta que facilite la realización de exámenes  en el área de inglés podría aportar en lo económico ya que se está automatizando dicho proceso; además, el prototipo podría ser escalable a otras instituciones, en especial, instituciones académicas donde su actividad principal gira alrededor del aprendizaje del lenguaje Inglés, permitiendo realizar exámenes a estudiantes sobre diferentes temas en esta área que se quieran validar.
\end{justify}\par

\end{document}