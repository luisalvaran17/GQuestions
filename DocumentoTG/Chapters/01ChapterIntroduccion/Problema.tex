\documentclass[../Main.tex]{subfiles}

\begin{document}
\section{Planteamiento del problema}

\subsection{Descripción del problema}

\begin{justify}
La copia en los exámenes es un problema que afecta a toda institución educativa, a pesar de las medidas tomadas, no se logra controlar, muchos estudiantes se ven en la necesidad de recurrir a esta práctica deshonesta con el fin de aprobar el curso \cite{1}.
\end{justify}\par

\begin{justify}
En el año 2011 un estudio realizado por Jara, Riascos y Romero \cite{2} sobre los exámenes ICFES del 2009, se obtuvo como resultado que alrededor del 96\% de las aulas hubo una pareja de estudiantes sospechosos de copia,  estos resultados fueron obtenidos a partir del análisis de todos los exámenes de ese año usando un algoritmo de detección de copias en exámenes, con el fin de comparar respuestas y detectar similitudes entre las pruebas resueltas.
\end{justify}\par

\begin{justify}
En los últimos años el avance tecnológico de alguna forma ha facilitado que los estudiantes hagan mal uso de este recurso para realizar copia en exámenes, además de prácticas convencionales. Cuando se buscan razones sobre la motivación de realizar copia por parte de los estudiantes en los exámenes, se encuentra que algunas son: falta de tiempo, falta de estudio, dificultad de la asignatura, ausencia de autoridad    por parte del docente, etc.
\end{justify}\par

\begin{justify}
Los experimentos realizados por Gino, Ayal y Ariely \cite{3} plantean que la deshonestidad académica puede ser influida por el individuo hacia otros individuos que se reconocen como parte del mismo círculo social, un comportamiento que se cumple dentro del contexto académico.
\end{justify}\par

\begin{justify}
En el estudio realizado por  Lina Martinez y Enrique Ramírez \cite{4} se logró obtener datos sobre la frecuencia con que estudiantes universitarios cometen fraudes académicos, a partir de encuestas realizadas a 3.300 estudiantes de cuatro universidades de Colombia entre los años 2003 y 2013, estos resultados mostraron que más del 90\% de encuestados admitieron haber cometido fraude; entre las conductas relacionadas se identificó que dejarse copiar en un examen e incluir a alguien en un grupo sin haber trabajo son de las más comunes.
\end{justify}\par

\begin{justify}
La Universidad representa el modelo a seguir en nuestra sociedad, por ende, es importante hallar soluciones que disminuyan la deshonestidad académica, además, los estudiantes que recurren a esta conducta tienen una alta probabilidad de comportarse igual en su vida profesional.
\end{justify}\par


\end{document}