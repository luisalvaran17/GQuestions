\documentclass[../Main.tex]{subfiles}
\begin{document}

\begin{justify}
    El presente trabajo trata sobre la problemática de deshonestidad académica «fraudes en exámenes» por parte de los estudiantes presentada en instituciones educativas, específicamente en universidades, se busca abordar la problemática desde la aplicación de la Ingeniería de Sistemas. Para ello se pretende la utilización de técnicas de procesamiento del lenguaje natural (NLP) que permitan generar exámenes a partir de algoritmos que utilizan estas técnicas. NLP corresponde a un área de la Inteligencia Artificial que consiste en el estudio de la interacción de las computadoras y el lenguaje humano,  la máquina aprende la sintaxis y el significado del lenguaje humano, lo procesa y le da la salida al usuario.
\end{justify}

\begin{justify}
    Los exámenes generados a partir del algoritmo adaptado para este propósito serán aplicados en el área de Inglés de la Universidad del Valle sede Tuluá, a través del desarrollo de un prototipo web que sigue el proceso correspondiente de desarrollo de software con metodologías ágiles.
\end{justify}

\end{document}