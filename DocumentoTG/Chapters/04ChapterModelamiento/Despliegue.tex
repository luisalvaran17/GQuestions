\documentclass[../Main.tex]{subfiles}
\begin{document}

    \section{Despliegue}
    \begin{justify}
    En el despliegue se tuvieron en cuenta varias alternativas de servicios Cloud Computing, entre las que están Google Cloud Platform, IBM Watson, Amazon Web Services (AWS) y Microsoft Azure, todos estos servicios son de pago, sin embargo, Google Cloud ofrece tres meses de prueba con 300 dólares y por esta razón se eligió esta plataforma para desplegar el modelo de Generación de preguntas y respuestas.
    
    Google Cloud platform (GCP) es una plataforma en la nube que permite crear, probar e implementar soluciones utilizando la infraestructura confiable y escalable de Google. GCP se ejecuta en la misma infraestructura que Google usa internamente para sus productos.
    \end{justify}
    
    \subsection{Modelos de prestación}
    \subsubsection{infraestructura como servicio (IaaS)}
        \begin{justify}
         Permite a TI ejecutar máquinas virtuales sin tener que invertir o administrar esta infraestructura informática ellos mismos. 
        \end{justify}
    
    \subsubsection{Plataforma como servicio (PaaS)}
        \begin{justify}
         Basado en el modelo IaaS. Los clientes optan por todos los beneficios de IaaS, además de obtener una infraestructura subyacente, como sistemas operativos y middleware.
        \end{justify}
        
    \subsubsection{Software como servicio (SaaS)}
        \begin{justify}
         Todo está disponible a través de la web: el proveedor aloja, administra y entrega toda la infraestructura, incluidas las aplicaciones. 
        \end{justify}
        
    \subsection{Productos de Google Cloud Platform}
    En esta sección se describe brevemente algunos productos que ofrece GCP.
    \subsubsection{IA y Aprendizaje automático}
        \begin{itemize}
            \item Vertex AI: Plataforma unificada para entrenar, alojar y administrar modelos de AA.
            \item Deep Learning VM Image: VM preconfiguradas para aprendizaje profundo.
            \item Contenedores Deep Learning: Contenedores previamente configurados para entornos de aprendizaje profundo
        \end{itemize}
    
    \subsubsection{Bases de datos}
    \begin{itemize}
        \item Cloud SQL: Base de datos completamente administrada para MySQL, PostgreSQL y SQL Server.
        \item Firebase Realtime Database: Base de datos NoSQL para almacenar y sincronizar datos en tiempo real.
    \end{itemize}
    
    \subsubsection{Procesamiento}
    \begin{itemize}
        \item App Engine: Plataforma de aplicaciones sin servidores para apps y backends.
        \item Compute Engine: Máquinas virtuales que se ejecutan en el centro de datos de Google.
        \item Cloud Run: Entorno completamente administrado para ejecutar apps alojadas en contenedores.
    \end{itemize}
    
    \subsubsection{Herramientas para desarrolladores}
    \begin{itemize}
        \item Artifact Registry: Almacena, administra y protege imágenes de contenedor y paquetes de lenguajes.
        \item Cloud Build: Plataforma de integración y entrega continuas.
    \end{itemize}
    
    \subsubsection{Almacenamiento}
    \begin{itemize}
        \item Archive Storage: Archivo de datos que ofrece velocidad de acceso en línea a muy bajo costo.
        \item Cloud Storage: Almacenamiento de objetos seguro, duradero y escalable.
        \item Filestore: Almacenamiento de archivos altamente escalable.
    \end{itemize}
    
    \subsection{Detalles de despliegue}
    \begin{justify}
    De los servicios descritos anteriormente se utilizaron Cloud Run, Cloud Build y Cloud Storage.
    
    Para realizar el despliegue en Cloud Run se tuvo que utilizar contenedores, específicamente Docker, al momento de compilar el servicio se verifica el archivo docker que hace referencia al proyecto.
    
    En los directorios del proyecto deben estar los archivos main.py, DockerFile, requirements.txt, .gcloudignore.
    
    La máquina virtual que ejecuta el algoritmo una vez está desplegado en Cloud Run utiliza CPU de 4 núcleos y 8GB, para utilizar GPU que sería lo ideal los cobros son altos y no es posible con la prueba gratuita; por esta razón la respuesta a las solicitudes API son tardías por limitaciones computacionales a pesar de estar corriendo en la nube (GCloud), sin embargo permitió tener el servicio disponible en la nube en vez de manera local y realizar peticiones HTTP utilizando el framework Flask.
    \end{justify}
    
\end{document}