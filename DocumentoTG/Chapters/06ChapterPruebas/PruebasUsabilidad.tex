\documentclass[../Main.tex]{subfiles}
\begin{document}

    \section{Pruebas de usabilidad}
    \begin{justify}
    Las pruebas de usabilidad o de UX son un método de prueba que permite medir qué tan amigable y fácil de usar es un software. Un pequeño grupo de usuario finales utiliza el software para exponer defectos de usabilidad.
    
    El propósito de realizar estas pruebas es para verificar diferentes aspectos importantes:
    
    \begin{itemize}
        \item Estética y diseño
        \item Efectividad del sistema
        \item Eficiencia
        \item Facilidad de uso
        \item Exactitud

    \end{itemize}
    
    Para llevar a cabo las pruebas de usabilidad se realizaron dos tipos de encuestas, el primer tipo de encuesta dirigido a un docente del curso Lectura de textos académicos en Inglés IV y el segundo tipo de encuesta dirigido a estudiantes de Inglés. Ambos tipos de encuestas comparten algunas preguntas o similares.

    \end{justify}
    
    \begin{justify}
    Trás realizar el análisis de los resultados de las pruebas de usabilidad realizadas se pudo observar los siguientes aspectos relevantes:
    
    \begin{enumerate}
        \item El diseño (apariencia) del prototipo web es altamente aceptado por los usuarios encuestados.
        \item La navegación en el prototipo web es muy buena y buena en un 80\% y 20\% respectivamente para los usuarios encuestados.
        \item Se identificó que el registro de usuarios se debe simplificar en cuanto a cantidad de campos necesarios para registrarse, sin embargo, al contar con el registro con cuenta de Google contrarresta la cantidad de campos que se deben rellenar.
        \item Las recomendaciones de inglés en el inicio del perfil estudiantes se debe mejorar, debido a la cantidad de recomendaciones o a la utilidad de este apartado.
        \item El diseño responsive y el uso del prototipo web en móviles es adecuado en todos los usuarios encuestados.
        \item El examen generado y presentado como ejemplo para la calificación de coherencia de estos por parte de los usuarios, es considerado como coherente o muy coherente; de otro modo, la dificultad del examen presentado se posiciona en un valor de 7 en una escala de menor a mayor (1-10).
        \item Todos los usuarios encuestados consideran que el prototipo web lo recomendarían y que el examen presentado como ejemplo es totalmente aplicable.
    \end{enumerate}
    \end{justify}
    
    \begin{justify}
        Las encuestas realizadas se encuentran en los anexos.
    \end{justify}

\end{document}

