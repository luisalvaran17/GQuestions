\documentclass[../Main.tex]{subfiles}
\begin{document}

\section{Conclusiones}

    \begin{justify}
    
    \begin{enumerate}
        \item Se logró evidenciar que llevar un proceso adecuado de desarrollo de software aumenta las probabilidades de éxito en términos de lograr los objetivos trazados y dar solución al problema o necesidad inicial. Una planificación correcta, el uso de metodologías ágiles, herramientas de gestión, controladores de versiones, entre otros aspectos, influyen directamente.

        \item El estado del arte Procesamiento de Lenguaje Natural se encuentra en un buen momento y en crecimiento acelerado, existen gran variedad de modelos sofisticados que con una buena integración dan solución a problemas del mundo real, como la generación de texto (Text Generation) y la respuesta a preguntas (Question Answering) en el proyecto.
            
        \item Se logró evidenciar que un software que genera exámenes de inglés utilizando tareas de Procesamiento de Lenguaje Natural puede ayudar a reducir el plagio en este tipo de exámenes, específicamente la copia entre estudiantes, ya que otros tipos de plagios podrían estar presentes, no obstante, el propósito principal del proyecto es lograr aportar en la reducción de los fraudes en exámenes de inglés. Este aporte a la reducción de plagios se logra en base a que, el prototipo web genera exámenes que son diferentes entre sí pues las preguntas están basadas en el texto que es único por examen, además, se tiene amplia variedad de temas para los textos, en efecto cada estudiante tiene un examen diferente.
        
        \item La utilización de modelos de Procesamiento de Lenguaje Natural implica altos costos computacionales y su despliegue puede suponer diferentes limitaciones especialmente de rendimiento, tiempos de respuesta, disponibilidad, usabilidad, entre otras.
        
        \item La realización de pruebas de software son indispensables, permitieron detectar errores que con una revisión humana no hubieran sido detectados a tiempo. Los casos de prueba fueron la base para la ejecución apropiada de pruebas, además, la automatización de éstas redujo considerablemente el tiempo que se fuera utilizado en sólo pruebas manuales. 
        
    \end{enumerate}
    
    \end{justify}

\section{Trabajos futuros}

    \begin{justify}
    Durante el desarrollo del proyecto se pudo identificar diferentes aspectos que complementarían al prototipo web:
    
    \begin{enumerate}
        \item Implementar más tipos de preguntas en la generación de preguntas, por ejemplo, completación, cloze, verdadero o falso, entre otras.
        
        \item Implementar la gamificación en el perfil estudiante y que a través de juegos trasladados a la parte educativa se pueda incentivar el conseguir mejores resultados, adquirir más conocimientos o mejorar en las habilidades del idioma Inglés.
        
        \item Mejorar los tiempos de respuesta cuando se generan exámenes a través de las peticiones API.
        
        \item Ampliar los temas de generación de exámenes a áreas de aprendizaje más específicas en el idioma Inglés.
    \end{enumerate}
    
    \end{justify}

\end{document}