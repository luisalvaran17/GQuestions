\documentclass[spanish, a4paper, 12pt, twoside, openany]{book} 

% -------------- Setup, do not change these ---------------
\usepackage{textcomp}
\usepackage[T1]{fontenc, url}
\usepackage[utf8]{inputenc}
\usepackage{titlesec}
\setcounter{secnumdepth}{4}
\usepackage{multirow}
\usepackage{minted} % Code highlighting
\usepackage{adjustbox}
\usepackage{graphicx}
\usepackage{rotating}
\usepackage{minitoc}
\usepackage{amsmath, amssymb, amsthm} % Mathematical packages
\usepackage{parskip} % Removing indenting in new paragraphs
\urlstyle{sf}
\usepackage{color}
\usepackage{subcaption} 
\usepackage{pdflscape}
\usepackage{appendix}
\usepackage{chngcntr} % needed for correct table numbering
\counterwithin{table}{section} % numbering of tables 
\counterwithin{figure}{section} % numbering of figures
\numberwithin{equation}{section} % numbering of equations
\hyphenpenalty=100000 % preventing splitting of words
\sloppy 
\raggedbottom 
\usepackage{xparse,nameref}
\usepackage[bottom]{footmisc} % Fotnotes are fixed to bottom of page
\usepackage{lipsum} % For genereating dummy text
\usepackage{amsmath}
\usepackage{color,fix-cm}
\usepackage{latexsym}
\usepackage{pict2e}
\usepackage{wasysym}
\usepackage{tikz}
\usepackage{comment}
% --------- You can edit from this point on --------


% ----- Appearance and language ----- 
\usepackage[spanish]{babel} % document language
\graphicspath{{Images/}{../Images/}} % path to images
\usepackage[margin=2.54cm]{geometry} % sets margins for the document
\usepackage{setspace}
\linespread{1.2} % line spread for the document
\usepackage{microtype}
\usepackage[hidelinks]{hyperref}
\usepackage{listings}
\usepackage{xcolor}

% ----- Sections -----
\titleformat*{\section}{\LARGE\bfseries} % \section heading
\titleformat*{\subsection}{\Large\bfseries} % \subsection heading
\titleformat*{\subsubsection}{\large\bfseries} % \subsubsection heading
% next three lines creates the \paragraph command with correct heading 
\titleformat{\paragraph}
{\normalfont\normalsize\bfseries}{\theparagraph}{1em}{}
\titlespacing*{\paragraph}
{0pt}{3.25ex plus 1ex minus .2ex}{1.5ex plus .2ex}


% ----- Figures and tables ----- 
\usepackage{fancyhdr}
\usepackage{subfiles}
\usepackage{array}
\usepackage[rightcaption]{sidecap}
\usepackage{wrapfig}
\usepackage{float}
\usepackage[labelfont=bf]{caption} % bold text for captions
\usepackage[para]{threeparttable} % fancy tables, check these before you use them
\usepackage{url}
\usepackage{hhline}

\usepackage{blindtext}

% ----- Sources -----
\bibliographystyle{apaeng} % citation and reference list style
\def\biblio{\clearpage\bibliographystyle{apaeng}\bibliography{References.bib}} % defines the \biblio command used for referencing in subfiles - DO NOT CHANGE


% ----- Header and footer -----
\pagestyle{fancy}
\fancyhead[RO,LE]{\thepage} % page number on right for odd pages and left for even pages in the header
\fancyhead[RE,LO]{\nouppercase{\rightmark}} % chapter name and number on the right for even pages and left for odd pages in the header
%\renewcommand{\headrulewidth}{0pt} % sets thickness of header line
\fancyfoot{} % removes page number on bottom of page


% ----- Header of the frontpage ----- 
\fancypagestyle{frontpage}{
	\fancyhf{}
	\renewcommand{\headrulewidth}{0pt}
	\renewcommand{\footrulewidth}{0pt}
	\vspace*{1\baselineskip}
	
	\fancyhead[C]{ \includegraphics[width=1in]{image1.png}} % 
}
\usepackage[
backend=biber,
style=numeric,
sorting=none
]{biblatex} %Imports biblatex package
\usepackage{pdfpages}

\addbibresource{References.bib} % adds the references to the document
% ----- Document starts here ----- 
\begin{document}
\def\biblio{} % resets the biblio command, if not here a new reference list will be produced after every chapter
\dominitoc % Initialization

%\pagenumbering{gobble} % suppress page numbering
%\thispagestyle{empty} % suppress header

%------------------------------------- PORTADA --------------------------------------------
\subfile{Chapters/00Chapter/Portadas}

%------------------------------------- JURADOS --------------------------------------------
%\newpage
\newpage
\phantomsection
\subfile{Chapters/00Chapter/Jurados}

\restoregeometry % restores the margins after frontpage
%\nocite{*} % uncomment if you want all sources to be printed in the reference list, including the ones which are not cited in the text 


\pagenumbering{roman} % starting roman page numbering
%\setcounter{page}{4} % sets pagecounter to 1
%------------------------------------- AGRADECIMIENTOS --------------------------------------------
\newpage
\phantomsection
\addcontentsline{toc}{section}{Agradecimientos}
\section*{Agradecimientos}
\subfile{Chapters/00Chapter/Agradecimientos}

%------------------------------------- TABLA DE CONTENIDO ----------------------------------------
\newpage
\phantomsection
{\setstretch{1.0} % line spacing for the list
\addcontentsline{toc}{section}{Índice general}
\tableofcontents
}

{\setstretch{1.0} 
\phantomsection
\phantomsection
\addcontentsline{toc}{section}{Índice de figuras}
\listoffigures}

{\setstretch{1.0} 
\renewcommand{\listtablename}{Índice de tablas}
\phantomsection
\addcontentsline{toc}{section}{Índice de tablas}
\listoftables}

%------------------------------------- RESUMEN --------------------------------------------
\newpage
\phantomsection
\thispagestyle{empty} % suppress header
\addcontentsline{toc}{section}{Abstract}
\section*{Abstract}
\subfile{Chapters/00Chapter/Abstract}

\newpage
\phantomsection
\thispagestyle{empty} % suppress header
\addcontentsline{toc}{section}{Resumen}
\section*{Resumen}
\subfile{Chapters/00Chapter/Resumen}

%------------------------------------- INTRODUCCION --------------------------------------------
\newpage
\phantomsection
\thispagestyle{empty} % suppress header
\addcontentsline{toc}{section}{Introducción}
\section*{Introducción}
\subfile{Chapters/01ChapterIntroduccion/Introduccion}

\newpage
\addtocontents{toc}{\protect\setcounter{tocdepth}{4}} % sets depth of toc to 4, 1.1.1.1
\pagenumbering{arabic} % Starting arabic page numbering
%\setcounter{page}{1} % sets pagecounter to 1

%------------------------------Capitulo UNO: Introducción-------------------------------------------
\chapter{Contexto y objetivos}
%\minitoc% Creating an actual minitoc
%\clearpage % clears the page after the chapter is finished

%Planteamiento del problema
\subfile{Chapters/01ChapterIntroduccion/Problema}

%Formulación del problema
\subfile{Chapters/01ChapterIntroduccion/Formulacion}

%Objetivos
\subfile{Chapters/01ChapterIntroduccion/Objetivos}

%Antecedentes
\subfile{Chapters/01ChapterIntroduccion/Antecedentes}

%Justificacion
\subfile{Chapters/01ChapterIntroduccion/Justificacion}

%Resultados esperados
\subfile{Chapters/01ChapterIntroduccion/ResultadosEsperados}

%Informacion capitulos
\subfile{Chapters/01ChapterIntroduccion/InformacionCapitulos}

%----------------------------- Capitulo DOS: Marco referencial --------------------------------
% New Chapter
\chapter{Marco referencial}
%\minitoc% Creating an actual minitoc
%\clearpage

%Marco de antecedentes
\subfile{Chapters/02ChapterMarco/MarcoAntecentes}

%Marco teorico
\subfile{Chapters/02ChapterMarco/MarcoTeorico}

%Marco conceptual
\subfile{Chapters/02ChapterMarco/MarcoConceptual}

%----------------------------- Capitulo TRES: Recoleccion --------------------------------
% New Chapter
\chapter{Recolección de información}
%\minitoc% Creating an actual minitoc
%\clearpage
\subfile{Chapters/03ChapterRecoleccion/03Recoleccion}

%----------------------------- Capitulo CUATRO: Modelamiento --------------------------------
\chapter{Modelamiento}
%\minitoc% Creating an actual minitoc
\begin{justify}
En este capítulo se presenta detalles sobre los algoritmos seleccionados y mencionados en el capítulo 3, detalles como preparación de datos, entrenamiento, evaluación, implementación, despliegue, entre otros.

La palabra algoritmo, modelo y sistema en este capítulo se utilizan indistintamente.
\end{justify}

%Generacion texto
\subfile{Chapters/04ChapterModelamiento/GeneracionTexto}

\section{Preguntas}
\begin{justify}
En esta sección se describe aspectos importantes del algoritmo de generación de preguntas seleccionado. 

El algoritmo de código abierto seleccionado \cite{45} es capaz de realizar dos tareas (generación de preguntas y evaluación de preguntas para mayor calidad), en la sección 4.2.1.1 se describe el generador de preguntas y el evaluador de preguntas, a pesar de presentarse por separado, es importante denotar que ambas tareas se encuentran integradas en el algoritmo gracias a Adam Montgomerie, Ingeniero de Machine Learning; esto a través de dos modelos entrenados y desplegados en Hugging Face Transformers construidos por Adam Montgomerie.
\end{justify}

%Generacion preguntas
\subfile{Chapters/04ChapterModelamiento/GeneracionPreguntas}

%Adaptamiento de modelos
\subfile{Chapters/04ChapterModelamiento/AdaptacionModelos}

%Despliegue de modelos
\subfile{Chapters/04ChapterModelamiento/Despliegue}
\clearpage

%----------------------------- Capitulo CINCO: Desarrollo web ------------------------------
\chapter{Desarrollo del prototipo web}
%\minitoc% Creating an actual minitoc

\begin{justify}
En este capítulo se presenta todo el proceso de desarrollo del prototipo web con los artefactos de software, además, se hace referencia a la metodología de desarrollo, las tecnologías usadas, el diseño, los detalles de la implementación y otras consideraciones importantes.  
\end{justify}

%Ingenieria
\subfile{Chapters/05ChapterDesarrollo/Ingenieria}

%Descripcion del sistema
\subfile{Chapters/05ChapterDesarrollo/DescripcionSistema}

%Implementacion
\subfile{Chapters/05ChapterDesarrollo/Implementacion}
\clearpage

%----------------------------- Capitulo SEIS: Pruebas  --------------------------------
\chapter{Pruebas}
%\minitoc% Creating an actual minitoc
\begin{justify}
En este capítulo se presentan las pruebas realizadas al Prototipo Web GQuestions, en el backend se realizaron pruebas unitarias con Django TestCase y en el frontend se realizaron pruebas de regresión con Selenium WebDriver para comprobar el correcto funcionamiento. También se presentan las pruebas de usabilidad que se realizaron a través de encuestas y los resultados obtenidos.
\end{justify}

%Pruebas unitarias
\subfile{Chapters/06ChapterPruebas/PruebasUnitarias}


%Pruebas de regresión
\subfile{Chapters/06ChapterPruebas/PruebasRegresion}

%Pruebas de usabilidad
\subfile{Chapters/06ChapterPruebas/PruebasUsabilidad}
\clearpage

%----------------------------- Capitulo SIETE: Conclusiones --------------------------------
\chapter{Conclusiones y trabajos futuros}
    \subfile{Chapters/07ChapterFinal/07Conclusion_FutureWork}
\clearpage

%---------------------------------- REFERENCIAS ----------------------------------------------
\newpage
\phantomsection
\renewcommand\refname{References} % name for the reference list
{\setstretch{1.4}% linespacing for the references
\addcontentsline{toc}{chapter}{Referencias} % to change the name of the references in the TOC
\printbibliography %Prints bibliography
}

%------------------------------------- ANEXOS ----------------------------------------------
\newpage
\phantomsection
\thispagestyle{empty} % suppress header
\addcontentsline{toc}{chapter}{Anexos} % to change the name of the references in the TOC
\section*{Anexos}
\subfile{Annexes/Anexos}

\thispagestyle{empty} % suppress header
\end{document}